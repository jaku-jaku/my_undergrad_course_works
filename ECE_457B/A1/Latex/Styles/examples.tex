
%%%%%%%%%%%%%%%%%%%%%%%%%%%%%%%%%%%%%%%%%%%%%%%%%%%%%%%%%%%%%%%%%%%%%%%%%%%%%%%%
%% ************************************************************************** %%
%% *                               Samples                                  * %%
%% ************************************************************************** %%
%%%%%%%%%%%%%%%%%%%%%%%%%%%%%%%%%%%%%%%%%%%%%%%%%%%%%%%%%%%%%%%%%%%%%%%%%%%%%%%%
%% - TABLE -- BEGIN --------------- %%
% \begin{table}[ht!]
%     \caption{Drainage time vs length of pipe} 
%     \label{Table:experimental_data}
%     \centering
%     \begin{tabular}{ |C{5cm}|C{2cm}|C{2cm}|C{2cm}|C{2cm}|}
%         \hline
%         & \textbf{Group 1} & \textbf{Group 2}  & \textbf{Group 3}  & \textbf{Group 4} \\
%         \hline
%         \textbf{Length of pipe [mm]}& 200 & 150 & 100 & 50\\
%         \hline
%         & \multicolumn{4}{|c|}{\textbf{Drainage time [sec]}} \\
%         \hline
%         \textbf{Trial 1} & 95 & 79.61 & 73.88 & 69.01 \\
%         \hline 
%         \textbf{Trial 2} & 94.58 & 79.03 & 73.78 & 69.84\\
%         \hline 
%         \textbf{Trial 3} & 94.48 & 79.08 & 74.73 & 69.44 \\
%         \hline 
%         \textbf{Average time [sec]} & 94.69 & 79.24 & 74.13 & 69.43\\
%         \hline
%         \textbf{Standard deviation [sec]} & 0.28 & 0.32 & 0.52 & 0.42 \\
%         \hline
%     \end{tabular}
% \end{table}
%% - TABLE -- END ---------------- %%

%% - FIG -- BEGIN --------------- %%
% \begin{figure}[H] 
%     \centering
%     \includegraphics[width=1\columnwidth]{Figs/bottle_sketch.png}
%     \caption{Sketch and dimension of the bottle model}
%     \label{fig:system_sketch}
% \end{figure}
%% - FIG -- END ----------------- %%

%% - LIST -- BEGIN --------------- %%
% \begin{enumerate}
%     \item Analysis is performed in 2D.
%     \item The diameter of the bottle is constant for the height of water considered.
%     \item The pipe is assumed to be straight with a uniform diameter. Hence, all minor losses are due to pipe protruding inside the bottle (re-entrant inlet). $K_{minor} = 0.5$ \cite{book:fluid_text_book}
%     \item Streamlines inside the bottle are assumed to be frictionless.
%     \item The starting flow is assumed to be turbulent with an $Re = 4000$ (will be refined in an iterative method).
%     \item The control volume takes into account the bottle and the pipe
%     \item The pipe is assumed to be parallel to the ground throughout the experiment (neglecting the deflection)
%     \item The origin is located at the pipe inlet($H_3$ above the ground since the pipe is horizontal)
%     \item The absolute roughness, $\varepsilon$, of the pipe is $0.0015mm$ \cite{book:fluid_text_book}.
%     \item Laminar flow is assumed if $Re < 2300$ \cite{book:fluid_text_book}.
%     \item Turbulent flow is assumed if $Re > 4000$ \cite{book:fluid_text_book}.
% \end{enumerate}
%% - LIST -- END ----------------- %%

%% - MATH -- BEGIN --------------- %%
% \begin{align}
%     \label{eqn:mass_conserv}
% 	& \dv{\iiint_{\forall_{bottle}}{\rho \dd{\forall_{bottle}}}}{t} + \iint_\mathbf{A} \rho (\Vec{v}\cdot\hat{n} \dd{\mathbf{A}_{pipe}})  = 0 
% \end{align}
%% - MATH -- END ----------------- %%

%% - CODE -- BEGIN --------------- %%
% \section{Code for Helper Class}
% \label{app:helper}
% \lstinputlisting[language=MATLAB, caption=Helper and commonly used functions by main, label=code:helper]{Code/helper.m}
%% - CODE -- END --------------- %%

